\documentclass[12pt]{article}
\usepackage{amsmath}
\pagestyle{empty}
\title{Chapter 1 - Energy in Thermal Physics}
\date{October 13, 2023}
\parindent 0px
\begin{document}
\maketitle
\section*{Problem 1.1}

The Fahrenheit temperature scale is defined so that ice melts at
32F and water boils at 212F.

(a) Derive the formulas for converting from Fahrenheit to Celsius and back.
(b) What is absolute zero on the Fahrenheit scale?

Actually derive it instead of reiterating stated facts.

\section*{GPT's Answer to Problem 1.1}

The Fahrenheit temperature scale is defined so that ice melts at \(32^\circ F\) and water boils at \(212^\circ F\).

\subsection*{(a) Deriving Formulas for Temperature Conversion}

Given: 
\begin{itemize}
    \item The melting point of ice is \(32^\circ F\) and \(0^\circ C\).
    \item The boiling point of water is \(212^\circ F\) and \(100^\circ C\).
\end{itemize}

Let's define two linear functions:
\begin{equation}
    C(m) = a \cdot m + b \quad \text{(i)}
\end{equation}
where \(C\) is the temperature in Celsius and \(m\) is the temperature in Fahrenheit, and
\begin{equation}
    F(c) = p \cdot c + q \quad \text{(ii)}
\end{equation}
where \(F\) is the temperature in Fahrenheit and \(c\) is the temperature in Celsius.

We can find \(a\) and \(b\) (and later \(p\) and \(q\)) by using the two given points in the Celsius and Fahrenheit scales.

1. \(C(32) = 0\) leads to:
\begin{equation}
    0 = 32a + b \quad \text{(iii)}
\end{equation}

2. \(C(212) = 100\) leads to:
\begin{equation}
    100 = 212a + b \quad \text{(iv)}
\end{equation}

To find the value of \(a\) we subtract equation (iii) from (iv):
\begin{align*}
    100 - 0 &= 212a - 32a \\
    100 &= 180a \\
    a &= \frac{100}{180} = \frac{5}{9}
\end{align*}

Now we find \(b\) using equation (iii) and the value of \(a\):
\begin{align*}
    0 &= 32a + b \\
    b &= -32a \\
    b &= -32 \times \frac{5}{9}
\end{align*}

So, 
\begin{equation}
    C(m) = \frac{5}{9}m - \frac{160}{9} \quad \text{(v)}
\end{equation}

Now for the Fahrenheit formula in terms of Celsius, using similar steps, we derive:
\begin{equation}
    F(c) = \frac{9}{5}c + 32 \quad \text{(vi)}
\end{equation}

\subsection*{(b) Absolute Zero on the Fahrenheit Scale}

Absolute zero is the lowest possible temperature where nothing could be colder and no heat energy remains in a substance. It is defined as \(-273.15^\circ C\) in the Celsius scale.

To find absolute zero on the Fahrenheit scale, we use equation (vi):
\begin{align*}
    F(-273.15) &= \frac{9}{5} \times (-273.15) + 32 \\
    F(-273.15) &= -491.67
\end{align*}

So, absolute zero is \(-491.67^\circ F\) on the Fahrenheit scale.

\subsection*{Comments}

GPT4 gives the correct answers and explanations.

\section*{Problem 1.2}

The Rankine temperature scale (abbreviated $^\circ R$) uses the same
size degrees as Fahrenheit, but measured up from absolute zero like kelvin (so
Rankine is to Fahrenheit as kelvin is to Celsius). Find the conversion formula
between Rankine and Fahrenheit, and also between Rankine and kelvin. What is
room temperature on the Rankine scale? Actually derive it instead of reiterating stated facts.

\section*{GPT's Answer to Problem 1.1}


The Rankine temperature scale (denoted as $^\circ R$) uses the same size degrees as Fahrenheit, but measured up from absolute zero like Kelvin (so Rankine is to Fahrenheit as Kelvin is to Celsius).

\subsection*{Conversion between Rankine and Fahrenheit}

We know that:
\begin{itemize}
    \item Absolute zero is $-491.67^\circ F$.
\end{itemize}

Given that both Fahrenheit and Rankine use the same interval size, the conversion formula from Fahrenheit (F) to Rankine (R) is found by adjusting the zero point from the Fahrenheit scale to absolute zero.

\begin{equation}
    R = F + 491.67 \quad \text{(i)}
\end{equation}

And conversely:
\begin{equation}
    F = R - 491.67 \quad \text{(ii)}
\end{equation}

\subsection*{Conversion between Rankine and Kelvin}

Now let's determine the conversion between the Kelvin (K) and Rankine (R) scales.

\begin{itemize}
    \item Absolute zero is $0K$ and also $0^\circ R$.
    \item The boiling point of water is $373.15K$ and $671.67^\circ R$ (using the conversion from Celsius to Kelvin and Fahrenheit to Rankine).
\end{itemize}

Let's define:
\begin{equation}
    K = m \cdot R + n \quad \text{(iii)}
\end{equation}
where K is the temperature in Kelvin and R is the temperature in Rankine, and m and n are constants to be determined.

Using the two points above, we can form two simultaneous equations:
\begin{enumerate}
    \item \(K(0) = 0\) implies:
    \begin{equation}
        0 = m \cdot 0 + n \implies n = 0 \quad \text{(iv)}
    \end{equation}
    \item \(K(671.67) = 373.15\) implies:
    \begin{equation}
        373.15 = 671.67 \cdot m + n \quad \text{(v)}
    \end{equation}
\end{enumerate}

Solving these two equations gives:
\[ m = \frac{373.15}{671.67} \]

So,
\begin{equation}
    K = \frac{373.15}{671.67} \cdot R \quad \text{(vi)}
\end{equation}

\subsection*{Room Temperature on the Rankine Scale}

Typically, room temperature is around $20^\circ C$ or $68^\circ F$.

Using the conversion from Fahrenheit to Rankine (equation i):
\[ R = 68 + 491.67 \]
\[ R = 559.67^\circ R \]

\subsection*{Comments}

GPT4 gives the correct answers and explanations. However, it chooses unwieldy number as the slope for Rankine to Kelvin equation. Upon reiteration which asks to simplify the formula, it does give the more well known form: $$K = \frac{5}{9}R$$.

\section*{Problem 1.3}

Determine the kelvin temperature for each of the following:

\begin{enumerate}
    \item[(a)] Human body temperature;
    \item[(b)] The boiling point of water (at the standard pressure of 1 atm);
    \item[(c)] The coldest day you can remember;
    \item[(d)] The boiling point of liquid nitrogen ($-196^\circ$C);
    \item[(e)] The melting point of lead ($327^\circ$C).
\end{enumerate}

Actually derive it instead of reiterating stated facts

\section*{GPT's Answer to Problem 1.3}

To convert temperatures from Celsius to Kelvin, we use the formula: $K = C + 273.15$. Applying this formula to each scenario:

\begin{enumerate}
    \item[(a)] The average human body temperature is approximately $37^\circ$C. \\
    So, $K = 37 + 273.15 = 310.15\,K$.

    \item[(b)] Water boils at $100^\circ$C under standard conditions. \\
    So, $K = 100 + 273.15 = 373.15\,K$.

    \item[(c)] Assuming a very cold day of $-30^\circ$C for this example. \\
    So, $K = -30 + 273.15 = 243.15\,K$.

    \item[(d)] Liquid nitrogen boils at $-196^\circ$C. \\
    So, $K = -196 + 273.15 = 77.15\,K$.

    \item[(e)] Lead has a melting point of $327^\circ$C. \\
    So, $K = 327 + 273.15 = 600.15\,K$.
\end{enumerate}

\subsection*{Comments}

GPT4 gives the correct answers and explanations.

\section*{Problem 1.4}

Does it ever make sense to say that one object is “twice as hot” as another? Does it matter whether one is referring to Celsius or kelvin temperatures? Explain.

\section*{GPT's Answer to Problem 1.4}

The concept of one object being "twice as hot" as another is not scientifically or thermodynamically sound when using the Celsius or Fahrenheit scales, because these scales have arbitrary zero points based on specific physical milestones (e.g., the freezing and boiling points of water for the Celsius scale). However, the Kelvin scale is an absolute temperature scale with its zero point at absolute zero, the theoretically coldest temperature and a natural zero indicating the absence of thermal energy.

To illustrate, consider two objects: 
\begin{itemize}
    \item Object A at $100^\circ$C
    \item Object B at $50^\circ$C
\end{itemize}

In the Celsius scale, stating that object A is "twice as hot" as object B is misleading because the scale's zero point is arbitrary, not representing a true absence of heat. 

However, when converted to the Kelvin scale (using $K = C + 273.15$), we have:
\begin{itemize}
    \item Object A: $100 + 273.15 = 373.15 K$
    \item Object B: $50 + 273.15 = 323.15 K$
\end{itemize}

Here, $373.15 K$ is not double $323.15 K$, showing that temperature ratios don't preserve when switching scales. This issue arises from the arbitrary zero points in the Celsius and Fahrenheit scales.

Conversely, for temperatures on the Kelvin scale:
\begin{itemize}
    \item Object C at $100 K$
    \item Object D at $200 K$
\end{itemize}

Object D has double the Kelvin temperature of Object C, meaning it has twice the thermal energy. This comparison is valid because the Kelvin scale's zero point is absolute zero, a natural and absolute reference point indicating no thermal motion.

In conclusion, the validity of the statement that one object is "twice as hot" as another depends on the temperature scale used. It doesn't hold for Celsius or Fahrenheit, due to their arbitrary zero points, but it does for Kelvin, as this scale starts at absolute zero, allowing meaningful comparisons of thermal energy ratios.

\subsection*{Comments}

\end{document}
