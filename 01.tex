\documentclass[12pt]{article}
\usepackage{amsmath}
\usepackage{gensymb}
\usepackage{amssymb}
\usepackage{siunitx} % Recommended for units handling
\pagestyle{empty}
\title{Chapter 1 - Energy in Thermal Physics}
\date{October 13, 2023}
\parindent 0px

\author{Ryan}

\begin{document}
\maketitle
\section*{Problem 1.1}

The Fahrenheit temperature scale is defined so that ice melts at
32F and water boils at 212F.

(a) Derive the formulas for converting from Fahrenheit to Celsius and back.
(b) What is absolute zero on the Fahrenheit scale?

Actually derive it instead of reiterating stated facts.

\section*{GPT's Answer to Problem 1.1}

The Fahrenheit temperature scale is defined so that ice melts at \(32^\circ F\) and water boils at \(212^\circ F\).

\subsection*{(a) Deriving Formulas for Temperature Conversion}

Given: 
\begin{itemize}
    \item The melting point of ice is \(32^\circ F\) and \(0^\circ C\).
    \item The boiling point of water is \(212^\circ F\) and \(100^\circ C\).
\end{itemize}

Let's define two linear functions:
\begin{equation}
    C(m) = a \cdot m + b \quad \text{(i)}
\end{equation}
where \(C\) is the temperature in Celsius and \(m\) is the temperature in Fahrenheit, and
\begin{equation}
    F(c) = p \cdot c + q \quad \text{(ii)}
\end{equation}
where \(F\) is the temperature in Fahrenheit and \(c\) is the temperature in Celsius.

We can find \(a\) and \(b\) (and later \(p\) and \(q\)) by using the two given points in the Celsius and Fahrenheit scales.

1. \(C(32) = 0\) leads to:
\begin{equation}
    0 = 32a + b \quad \text{(iii)}
\end{equation}

2. \(C(212) = 100\) leads to:
\begin{equation}
    100 = 212a + b \quad \text{(iv)}
\end{equation}

To find the value of \(a\) we subtract equation (iii) from (iv):
\begin{align*}
    100 - 0 &= 212a - 32a \\
    100 &= 180a \\
    a &= \frac{100}{180} = \frac{5}{9}
\end{align*}

Now we find \(b\) using equation (iii) and the value of \(a\):
\begin{align*}
    0 &= 32a + b \\
    b &= -32a \\
    b &= -32 \times \frac{5}{9}
\end{align*}

So, 
\begin{equation}
    C(m) = \frac{5}{9}m - \frac{160}{9} \quad \text{(v)}
\end{equation}

Now for the Fahrenheit formula in terms of Celsius, using similar steps, we derive:
\begin{equation}
    F(c) = \frac{9}{5}c + 32 \quad \text{(vi)}
\end{equation}

\subsection*{(b) Absolute Zero on the Fahrenheit Scale}

Absolute zero is the lowest possible temperature where nothing could be colder and no heat energy remains in a substance. It is defined as \(-273.15^\circ C\) in the Celsius scale.

To find absolute zero on the Fahrenheit scale, we use equation (vi):
\begin{align*}
    F(-273.15) &= \frac{9}{5} \times (-273.15) + 32 \\
    F(-273.15) &= -491.67
\end{align*}

So, absolute zero is \(-491.67^\circ F\) on the Fahrenheit scale.

\subsection*{Comments}

GPT4 gives the correct answers and explanations.

\section*{Problem 1.2}

The Rankine temperature scale (abbreviated $^\circ R$) uses the same
size degrees as Fahrenheit, but measured up from absolute zero like kelvin (so
Rankine is to Fahrenheit as kelvin is to Celsius). Find the conversion formula
between Rankine and Fahrenheit, and also between Rankine and kelvin. What is
room temperature on the Rankine scale? Actually derive it instead of reiterating stated facts.

\section*{GPT's Answer to Problem 1.1}


The Rankine temperature scale (denoted as $^\circ R$) uses the same size degrees as Fahrenheit, but measured up from absolute zero like Kelvin (so Rankine is to Fahrenheit as Kelvin is to Celsius).

\subsection*{Conversion between Rankine and Fahrenheit}

We know that:
\begin{itemize}
    \item Absolute zero is $-491.67^\circ F$.
\end{itemize}

Given that both Fahrenheit and Rankine use the same interval size, the conversion formula from Fahrenheit (F) to Rankine (R) is found by adjusting the zero point from the Fahrenheit scale to absolute zero.

\begin{equation}
    R = F + 491.67 \quad \text{(i)}
\end{equation}

And conversely:
\begin{equation}
    F = R - 491.67 \quad \text{(ii)}
\end{equation}

\subsection*{Conversion between Rankine and Kelvin}

Now let's determine the conversion between the Kelvin (K) and Rankine (R) scales.

\begin{itemize}
    \item Absolute zero is $0K$ and also $0^\circ R$.
    \item The boiling point of water is $373.15K$ and $671.67^\circ R$ (using the conversion from Celsius to Kelvin and Fahrenheit to Rankine).
\end{itemize}

Let's define:
\begin{equation}
    K = m \cdot R + n \quad \text{(iii)}
\end{equation}
where K is the temperature in Kelvin and R is the temperature in Rankine, and m and n are constants to be determined.

Using the two points above, we can form two simultaneous equations:
\begin{enumerate}
    \item \(K(0) = 0\) implies:
    \begin{equation}
        0 = m \cdot 0 + n \implies n = 0 \quad \text{(iv)}
    \end{equation}
    \item \(K(671.67) = 373.15\) implies:
    \begin{equation}
        373.15 = 671.67 \cdot m + n \quad \text{(v)}
    \end{equation}
\end{enumerate}

Solving these two equations gives:
\[ m = \frac{373.15}{671.67} \]

So,
\begin{equation}
    K = \frac{373.15}{671.67} \cdot R \quad \text{(vi)}
\end{equation}

\subsection*{Room Temperature on the Rankine Scale}

Typically, room temperature is around $20^\circ C$ or $68^\circ F$.

Using the conversion from Fahrenheit to Rankine (equation i):
\[ R = 68 + 491.67 \]
\[ R = 559.67^\circ R \]

\subsection*{Comments}

GPT4 gives the correct answers and explanations. However, it chooses unwieldy number as the slope for Rankine to Kelvin equation. Upon reiteration which asks to simplify the formula, it does give the more well known form: $$K = \frac{5}{9}R$$.

\section*{Problem 1.3}

Determine the kelvin temperature for each of the following:

\begin{enumerate}
    \item[(a)] Human body temperature;
    \item[(b)] The boiling point of water (at the standard pressure of 1 atm);
    \item[(c)] The coldest day you can remember;
    \item[(d)] The boiling point of liquid nitrogen ($-196^\circ$C);
    \item[(e)] The melting point of lead ($327^\circ$C).
\end{enumerate}

Actually derive it instead of reiterating stated facts

\section*{GPT's Answer to Problem 1.3}

To convert temperatures from Celsius to Kelvin, we use the formula: $K = C + 273.15$. Applying this formula to each scenario:

\begin{enumerate}
    \item[(a)] The average human body temperature is approximately $37^\circ$C. \\
    So, $K = 37 + 273.15 = 310.15\,K$.

    \item[(b)] Water boils at $100^\circ$C under standard conditions. \\
    So, $K = 100 + 273.15 = 373.15\,K$.

    \item[(c)] Assuming a very cold day of $-30^\circ$C for this example. \\
    So, $K = -30 + 273.15 = 243.15\,K$.

    \item[(d)] Liquid nitrogen boils at $-196^\circ$C. \\
    So, $K = -196 + 273.15 = 77.15\,K$.

    \item[(e)] Lead has a melting point of $327^\circ$C. \\
    So, $K = 327 + 273.15 = 600.15\,K$.
\end{enumerate}

\subsection*{Comments}

GPT4 gives the correct answers and explanations.

\section*{Problem 1.4}

Does it ever make sense to say that one object is “twice as hot” as another? Does it matter whether one is referring to Celsius or kelvin temperatures? Explain.

\section*{GPT's Answer to Problem 1.4}

The concept of one object being "twice as hot" as another is not scientifically or thermodynamically sound when using the Celsius or Fahrenheit scales, because these scales have arbitrary zero points based on specific physical milestones (e.g., the freezing and boiling points of water for the Celsius scale). However, the Kelvin scale is an absolute temperature scale with its zero point at absolute zero, the theoretically coldest temperature and a natural zero indicating the absence of thermal energy.

To illustrate, consider two objects: 
\begin{itemize}
    \item Object A at $100^\circ$C
    \item Object B at $50^\circ$C
\end{itemize}

In the Celsius scale, stating that object A is "twice as hot" as object B is misleading because the scale's zero point is arbitrary, not representing a true absence of heat. 

However, when converted to the Kelvin scale (using $K = C + 273.15$), we have:
\begin{itemize}
    \item Object A: $100 + 273.15 = 373.15 K$
    \item Object B: $50 + 273.15 = 323.15 K$
\end{itemize}

Here, $373.15 K$ is not double $323.15 K$, showing that temperature ratios don't preserve when switching scales. This issue arises from the arbitrary zero points in the Celsius and Fahrenheit scales.

Conversely, for temperatures on the Kelvin scale:
\begin{itemize}
    \item Object C at $100 K$
    \item Object D at $200 K$
\end{itemize}

Object D has double the Kelvin temperature of Object C, meaning it has twice the thermal energy. This comparison is valid because the Kelvin scale's zero point is absolute zero, a natural and absolute reference point indicating no thermal motion.

In conclusion, the validity of the statement that one object is "twice as hot" as another depends on the temperature scale used. It doesn't hold for Celsius or Fahrenheit, due to their arbitrary zero points, but it does for Kelvin, as this scale starts at absolute zero, allowing meaningful comparisons of thermal energy ratios.

\subsection*{Comments}

GPT4 gives the correct answers and explanations.

\section*{Problem 1.5}

When you’re sick with a fever and you take your temperature with a thermometer, approximately what is the relaxation time?

\section*{GPT's Answer to Problem 1.5}

The relaxation time refers to the duration required for the thermometer to reach thermal equilibrium with the body, thus accurately indicating the body's temperature. This period is associated with the thermal time constant, which is linked to the thermometer's thermal mass, the specific heat capacity of its material, and the heat transfer coefficient between the thermometer and the body.

The heat transfer can be represented by the differential equation:
\[
\frac{dT}{dt} = \frac{(T_{\text{body}} - T)}{\tau}
\]

where:
\begin{itemize}
    \item \( \frac{dT}{dt} \) is the rate of temperature change of the thermometer,
    \item \( T_{\text{body}} \) is the body’s constant temperature,
    \item \( T \) is the thermometer’s temperature at time \( t \),
    \item \( \tau \) is the system’s thermal time constant.
\end{itemize}

The solution to this equation is:
\[
T(t) = T_{\text{body}} + (T_{\text{initial}} - T_{\text{body}}) e^{-t/\tau}
\]

where:
\begin{itemize}
    \item \( T(t) \) is the thermometer’s temperature at time \( t \),
    \item \( T_{\text{initial}} \) is the thermometer’s temperature at \( t = 0 \),
    \item \( e \) is the base of the natural logarithm,
    \item \( t \) is time,
    \item \( \tau \) is the thermal time constant.
\end{itemize}

The relaxation time is approximately the time for the thermometer to be within a small percentage (e.g., 1\%) of the body's temperature, often around 3 to 5 time constants (\( 3\tau \) to \( 5\tau \)). However, the exact value for \( \tau \) and consequently the relaxation time cannot be calculated without specific details about the thermometer's physical properties and the heat transfer coefficient. Practically, relaxation times range from several seconds for advanced digital thermometers to a few minutes for traditional mercury or alcohol thermometers.

\subsection*{Comments}

GPT4 gives the correct answers and explanations. However, some of the concepts were too advanced for students with minimal understanding of chemistry. Here is the detailed explanation:

\section*{Understanding Thermal Concepts}

Let's simplify some scientific concepts to understand how a thermometer measures temperature:

\begin{enumerate}
  \item \textbf{Thermal Time Constant:} This term describes how fast a system (like a thermometer) responds to temperature changes. A smaller thermal time constant means the thermometer can read your temperature quickly, while a larger one means it takes longer.

  \item \textbf{Thermal Mass:} This is the thermometer's ability to absorb and store heat. A thermometer with high thermal mass needs more time to absorb heat, hence taking longer to display the correct temperature.

  \item \textbf{Specific Heat Capacity:} This property tells us how much heat a material (the substance making up the thermometer) needs to change its temperature. Materials with high specific heat capacities require more heat to increase in temperature.

  \item \textbf{Heat Transfer Coefficient:} This coefficient measures the efficiency of heat transfer between your body and the thermometer. Better heat transfer means a quicker and more accurate reading.
\end{enumerate}

Now, let's decipher the equation that represents heat transfer:

\[
\frac{dT}{dt} = \frac{(T_{\text{body}} - T)}{\tau}
\]

The terms represent:
\begin{itemize}
  \item \(\frac{dT}{dt}\): The rate of the thermometer’s temperature change. A higher value means a rapid change.
  
  \item \(T_{\text{body}}\): The constant temperature of your body.
  
  \item \(T\): The thermometer’s temperature at any moment, which rises from room temperature as it absorbs your body’s heat.
  
  \item \(\tau\): The thermal time constant, indicating how quickly the thermometer responds to temperature changes.
\end{itemize}

In essence, this equation says that the speed of the thermometer's temperature change depends on both the temperature difference between your body and the thermometer and on the thermometer's ability to adapt to temperature changes (represented by \(\tau\)).

\section*{Problem 1.6}

Give an example to illustrate why you cannot accurately judge the
temperature of an object by how hot or cold it feels to the touch.

\section*{GPT's Answer to Problem 1.6}

To comprehend why the human touch is an unreliable thermometer, we need to consider thermodynamics, heat transfer, and human physiology principles.

\begin{enumerate}
  \item \textbf{Subjectivity of Human Perception:} Our nervous system perceives temperature via thermal receptors in the skin. These receptors are sensitive to heat flow rather than absolute temperature, responding more to temperature changes than to constant temperatures.

  \item \textbf{Heat Transfer:} The sensation of heat or coldness comes from heat transfer between our hand and the object, not just the object's temperature. The rate of heat transfer depends on the temperature difference and the object's thermal conductivity.

Let's consider the heat transfer \( q \) based on Newton's Law of Cooling:
\[
q = hA(T_{\text{object}} - T_{\text{skin}})
\]
where:
\begin{itemize}
  \item \( q \) is the heat transfer rate,
  \item \( h \) is the heat transfer coefficient,
  \item \( A \) is the contact area,
  \item \( T_{\text{object}} \) is the object's temperature,
  \item \( T_{\text{skin}} \) is the skin's temperature, approximately 34°C.
\end{itemize}

Here, \( h \) is influenced by the material's thermal conductivity \( k \). High thermal conductivity means rapid heat transfer, leading to intense sensations of heat or coldness, even if the object's temperature isn't extreme.

For instance, metal feels colder than wood at the same temperature because metal's higher thermal conductivity pulls heat from your hand more quickly, which your receptors interpret as a colder sensation.

  \item \textbf{Adaptation:} Our sensory receptors adapt over time, becoming less responsive to unchanging stimuli. This is why an object doesn't feel as hot or cold after you've been holding it for some time.
\end{enumerate}

These insights demonstrate that our sensory system doesn't gauge actual temperature—instead, it interprets heat flow, a perception influenced by material properties, contact area, and individual physiological differences. Hence, we cannot accurately assess an object's true temperature by how hot or cold it feels to our touch.

\subsection*{Comments}

GPT4 gives the correct answers and explanations. However, the explanation could have been more clarified by giving more examples that explain the influence of different factors such as contact area. 

\section*{Problem 1.7}

When the temperature of liquid mercury increases by one degree Celsius (or one kelvin), its volume increases by one part in 5500 . The fractional increase in volume per unit change in temperature (when the pressure is held fixed) is called the thermal expansion coefficient, $\beta$ :
$$
\beta \equiv \frac{\Delta V / V}{\Delta T}
$$
(where $V$ is volume, $T$ is temperature, and $\Delta$ signifies a change, which in this case should really be infinitesimal if $\beta$ is to be well defined). So for mercury, $\beta=1 / 5500 \mathrm{~K}^{-1}=1.81 \times 10^{-4} \mathrm{~K}^{-1}$. (The exact value varies with temperature, but between $0^{\circ} \mathrm{C}$ and $200^{\circ} \mathrm{C}$ the variation is less than $1 \%$.)

(a) Get a mercury thermometer, estimate the size of the bulb at the bottom, and then estimate what the inside diameter of the tube has to be in order for the thermometer to work as required. Assume that the thermal expansion of the glass is negligible.

(b) The thermal expansion coefficient of water varies significantly with temperature: It is $7.5 \times 10^{-4} \mathrm{~K}^{-1}$ at $100^{\circ} \mathrm{C}$, but decreases as the temperature is lowered until it becomes zero at $4^{\circ} \mathrm{C}$. Below $4^{\circ} \mathrm{C}$ it is slightly negative, reaching a value of $-0.68 \times 10^{-4} \mathrm{~K}^{-1}$ at $0^{\circ} \mathrm{C}$. (This behavior is related to the fact that ice is less dense than water.) With this behavior in mind, imagine the process of a lake freezing over, and discuss in some detail how this process would be different if the thermal expansion coefficient of water were always positive.

\section*{GPT's Answer to Problem 1.7}

\noindent (a) The question involves determining the necessary inner diameter of the capillary tube in a mercury thermometer, given that the thermal expansion of mercury is the primary factor affecting the thermometer's readings. Assuming the bulb of a typical thermometer holds about 0.5 ml (or 0.5 cubic centimeters) of mercury — an estimated average — we can use the given thermal expansion coefficient of mercury to find out how much the volume of mercury expands per degree Celsius and then determine the necessary diameter of the tube.

For mercury, $\beta = \frac{1}{5500}$ per K (or per $\degree$C). If the volume of the mercury in the bulb is 0.5 cc, then for each degree Celsius increase in temperature, the volume of the mercury will increase by $(0.5 \, \text{cc}) \times (\frac{1}{5500}) = 9.09 \times 10^{-5} \, \text{cc}$.

This additional volume of mercury must rise into the capillary tube of the thermometer. The capillary tube is essentially a cylindrical shape, and the volume of a cylinder is given by $V = \pi r^2 h$, where $r$ is the radius and $h$ is the height (which will correspond to the increased length that the mercury column rises per degree Celsius, which is how we read the temperature).

If we want the thermometer to register a 1$\degree$C change for every millimeter that the mercury rises (which is a typical scale), the change in height ($h$) will be 1 mm = 0.1 cm. Therefore, we can solve for the radius ($r$) of the tube using the volume formula:

\[
9.09 \times 10^{-5} \, \text{cc} = \pi r^2 \times 0.1 \, \text{cm}
\]

Solving for $r$, we get approximately $r \approx 0.0054 \, \text{cm}$ or $0.054 \, \text{mm}$. Therefore, the inner diameter of the tube should be approximately twice this, or about $0.11 \, \text{mm}$. \\

\noindent (b) The unique property of water is that its density increases as it cools from $100\degree\mathrm{C}$ to $4\degree\mathrm{C}$, but below $4\degree\mathrm{C}$, it expands (its density decreases) until it freezes. This is due to the molecular structure of ice, which forms a hexagonal lattice that occupies more space than liquid water. Because of this, ice floats on water, and lakes freeze from the top down.

If the thermal expansion coefficient of water were always positive (meaning that water continued to contract upon cooling), water would become denser as it cooled, all the way until it freezes. In this scenario, colder water would always sink to the bottom, and lakes would freeze from the bottom up.

This would have profound ecological implications. Aquatic life relies on the bottom layer of the lake remaining liquid and warmer during the winter, a result of ice's insulating properties when it floats on the surface. If lakes froze from the bottom up, many aquatic species might not survive the cold temperatures.

Additionally, if ice didn't float, it would continue to accumulate at the bottom of bodies of water, potentially never melting entirely in some areas and leading to a "runaway" freezing effect, drastically lowering the water levels and making it much more difficult for temperatures to rise enough to melt the ice during warmer seasons. This phenomenon could have significantly altered the climate, geography, and biological makeup of the Earth's aquatic systems.

\subsection*{Comments}

GPT4 gives the correct answers and explanations.

\section*{Problem 1.8}

For a solid, we also define the linear thermal expansion coefficient, $\alpha$, as the fractional increase in length per degree:
\[
\alpha \equiv \frac{\Delta L / L}{\Delta T}
\]
(a) For steel, $\alpha$ is $1.1 \times 10^{-5} \, \mathrm{K}^{-1}$. Estimate the total variation in length of a 1-km steel bridge between a cold winter night and a hot summer day. \\
(b) The dial thermometer in Figure 1.2 uses a coiled metal strip made of two different metals laminated together. Explain how this works. \\
(c) Prove that the volume thermal expansion coefficient of a solid is equal to the sum of its linear expansion coefficients in the three directions: $\beta= \alpha_x+\alpha_y+\alpha_z$. (So for an isotropic solid, which expands the same in all directions, $\beta=3 \alpha$.) \\

\subsection*{GPT's Answer to Problem 1.8}

\subsubsection*{(a)}
Given the linear thermal expansion coefficient of steel, $\alpha = 1.1 \times 10^{-5} \, \mathrm{K}^{-1}$, we can estimate the change in length of a 1-km steel bridge due to temperature changes. 

Let's assume a temperature variation between a cold winter night and a hot summer day is approximately $40 \, \si{\degreeCelsius}$ (for example, from $-20 \, \si{\degreeCelsius}$ to $20 \, \si{\degreeCelsius}$). 

The change in length $\Delta L$ is given by the formula:
\[
\Delta L = \alpha \cdot L \cdot \Delta T
\]
where $L = 1000 \, \text{meters} = 1 \, \text{km}$ (original length of the bridge) and $\Delta T = 40 \, \si{\degreeCelsius}$ (change in temperature).

Plugging in the numbers:
\[
\Delta L = (1.1 \times 10^{-5} \, \mathrm{K}^{-1}) \cdot (1000 \, \text{m}) \cdot (40 \, \si{\degreeCelsius}) = 0.44 \, \text{m}
\]
So, the bridge would expand by 0.44 meters, or 44 centimeters, in the temperature range considered.

\subsubsection*{(b)}
The dial thermometer works based on the principle of thermal expansion and the fact that different metals expand at different rates. The coiled metal strip in the thermometer is made of two different metals laminated together. When there is a change in temperature, one metal in the strip expands more than the other, causing the strip to coil or uncoil. The greater the temperature change, the tighter or looser the coil becomes. This mechanical change in the coil is then translated into a rotational movement of the needle in the dial of the thermometer, providing a temperature reading.

\subsubsection*{(c)}
To prove that the volume thermal expansion coefficient $\beta$ of a solid is the sum of its linear expansion coefficients in the three directions ($\alpha_x, \alpha_y, \alpha_z$), we start by considering the initial volume of the solid $V = L_x \cdot L_y \cdot L_z$, where $L_x, L_y,$ and $L_z$ are the initial lengths in the three directions.

When the temperature changes by $\Delta T$, the new lengths will be $L_x + \Delta L_x, L_y + \Delta L_y,$ and $L_z + \Delta L_z$, where $\Delta L_x = \alpha_x \cdot L_x \cdot \Delta T$ and similarly for $\Delta L_y$ and $\Delta L_z$.

The new volume $V'$ is:
\[
V' = (L_x + \Delta L_x) \cdot (L_y + \Delta L_y) \cdot (L_z + \Delta L_z)
\]

Expanding and neglecting the higher-order terms ($\Delta L^2$ and $\Delta L^3$ because they are very small compared to the first order changes):
\[
V' \approx L_x \cdot L_y \cdot L_z + L_x \cdot L_y \cdot \Delta L_z + L_x \cdot \Delta L_y \cdot L_z + \Delta L_x \cdot L_y \cdot L_z
\]

So, the change in volume $\Delta V = V' - V$ is:
\[
\Delta V \approx L_x \cdot L_y \cdot \Delta L_z + L_x \cdot \Delta L_y \cdot L_z + \Delta L_x \cdot L_y \cdot L_z
\]

Substituting $\Delta L_x = \alpha_x \cdot L_x \cdot \Delta T$ and similarly for $y$ and $z$, we get:
\[
\Delta V \approx L_x \cdot L_y \cdot L_z \cdot (\alpha_x + \alpha_y + \alpha_z) \cdot \Delta T = V \cdot (\alpha_x + \alpha_y + \alpha_z) \cdot \Delta T
\]

The volume expansion coefficient $\beta$ is then:
\[
\beta = \frac{\Delta V / V}{\Delta T} = \alpha_x + \alpha_y + \alpha_z
\]

For an isotropic solid, $\alpha_x = \alpha_y = \alpha_z = \alpha$, so $\beta = 3\alpha$.

\subsection*{Comments}

GPT4 gives the correct answers and explanations.

\section*{Problem 1.9}

\textbf{Problem 1.9.} What is the volume of one mole of air, at room temperature and 1 atm pressure?

\subsection*{Solution}

To find the volume of one mole of air at room temperature and 1 atm pressure, we use the ideal gas law:
\[
PV = nRT
\]
where:
\begin{itemize}
  \item \(P\) is the pressure,
  \item \(V\) is the volume,
  \item \(n\) is the number of moles,
  \item \(R\) is the universal gas constant, and
  \item \(T\) is the temperature in kelvins.
\end{itemize}

Given that the pressure \(P = \SI{1}{atm}\), the number of moles \(n = 1\), and the universal gas constant \(R = \SI{8.31}{\joule\per\mole\per\kelvin}\), we need to convert the pressure from atmospheres to pascals (the SI unit) and the temperature from Celsius to kelvins (the SI unit). Room temperature is usually considered to be \(\SI{25}{\degreeCelsius}\), which is \(298\) K (\(25 + 273\)). The pressure of \(1\) atm is equivalent to \(\SI{101325}{\pascal}\).

Plugging in the values, we have:
\[
(\SI{101325}{\pascal}) \cdot V = (1) \cdot (\SI{8.31}{\joule\per\mole\per\kelvin}) \cdot (298 \, \text{K})
\]
Solving for \(V\), we find:
\[
V \approx \SI{0.024}{\meter\cubed} \, \text{or} \, \SI{24}{\liter}
\]

So, the volume of one mole of air at room temperature and \(1\) atm pressure is approximately \(\SI{24}{\liter}\).


\end{document}
